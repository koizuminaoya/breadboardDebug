\documentclass[a4paper, landscape, 12pt]{ltjsarticle}
\usepackage{luatexja} %% https://ja.overleaf.com/learn/latex/Japanese#luatex-ja_package_bundle_with_LuaLaTeX

\usepackage[top=20mm, bottom=10mm, left=10mm, right=10mm]{geometry} % 余白設定
\usepackage{tikz}
\usetikzlibrary{arrows.meta, positioning, calc, shapes, shadows}
\usepackage{anyfontsize} % フォントサイズを自在に変える
\usepackage{comment} % コメント用

% ▼▼▼ この1行でページ番号を消します ▼▼▼
\pagestyle{empty}

%%
%% ここから本文
%% 
\begin{document}


% --- タイトル部分 ---
\begin{center}
    {\fontsize{24pt}{0pt}\selectfont \textbf{特性要因図:ブレッドボードの回路が動かない}} \\[10pt]
    
    % --- 説明文 ---
    \begin{minipage}{0.9\textwidth}
        \large
        \textbf{【デバッグの心得】} \\
        回路が動かないときは、闇雲に配線を触るのではなく、\textbf{「原因の可能性が高い順(左側の①)」}に、一つずつチェックしていくことが解決への近道です。
        「たぶん大丈夫」という思い込みを捨てて、コツコツと確認しながら進めましょう。
    \end{minipage}
\end{center}

\vspace{2mm}

% --- 特性要因図本体 ---
\begin{tikzpicture}[
    % デザイン定義
    % 背骨 (メインの矢印)
    spine/.style={
        -Stealth,          % 矢印の形
        line width=4pt, 
        color=black!90, 
        >={Stealth[length=12mm]} % 矢印の頭を大きく
    },
    % 大骨 (カテゴリーの矢印):背骨に向かう
    rib/.style={
        Stealth-,          % 始点(背骨側)に矢印をつける
        line width=2.5pt, 
        color=black!70,
        >={Stealth[length=8mm]}
    },
    % 小骨 (各項目の矢印):大骨に向かう
    subrib/.style={
        Stealth-,          % 始点(大骨側)に矢印をつける
        line width=1.5pt, 
        color=black!50,
        >={Stealth[length=5mm]}
    },    
    % カテゴリーラベル
    catlabel/.style={
        font=\bfseries\huge, 
        align=center, 
        fill=white, 
        draw=black!90, 
        line width=1.5pt,
        rounded corners=5pt,
        inner sep=8pt,
        drop shadow
    },
    % チェック項目
    item/.style={font=\bfseries, anchor=west}
]

\begin{comment}

\begin{tikzpicture}[
    % デザイン定義
    spine/.style={->, line width=2pt, >={Stealth[length=5mm]}},
    rib/.style={line width=1.5pt, color=gray!80},
    subrib/.style={-, line width=1pt},
    label/.style={font=\bfseries, align=center},
    item/.style={font=\small, anchor=west}
]
\end{comment}

    % --- 背骨(メインストリーム) ---
    \draw[spine] (0, 0) -- (20, 0) node[right, draw, text width=5cm, align=center, rounded corners] {\Large 回路が\\動かない};

    % --- 大骨(カテゴリー)の設定 ---
    % 上側
    \coordinate (c1) at (3, 0);     \coordinate (top1) at (1,   5);
    \coordinate (c2) at (11, 0);    \coordinate (top2) at (9,   5);
    \coordinate (c5) at (19, 0);    \coordinate (top3) at (17,  5);
    
    % 下側
    \coordinate (c3) at (5, 0);     \coordinate (btm1) at (3,  -5);
    \coordinate (c4) at (13, 0);    \coordinate (btm2) at (11, -5);

    % --- 描画と項目の追加 ---

    % =============================================
    % ① 電源・GND(左上) - 最優先チェック
    % =============================================
%    \draw[rib] (c1) -- (top1) node[above, label] {① 電源・GND};
    \draw[rib] (c1) -- (top1) node[above, catlabel] {① 電源・GND};
    \draw[subrib] ($(c1)!0.8!(top1)$) -- ++(1.5, 0) node[item] {電池切れ・スイッチ入れ忘れ};
    \draw[subrib] ($(c1)!0.6!(top1)$) -- ++(1.5, 0) node[item] {極性(+−)逆};
    \draw[subrib] ($(c1)!0.4!(top1)$) -- ++(1.5, 0) node[item] {GND共通化忘れ};     \draw[subrib] ($(c1)!0.2!(top1)$) -- ++(1.5, 0) node[item] {供給電圧が違う};


    % 2. 配線・接触(左下)
    \draw[rib] (c3) -- (btm1) node[below, catlabel] {② 配線・接触};
    \draw[subrib] ($(c3)!0.8!(btm1)$) -- ++(1.5, 0) node[item] {穴の1列ずれ};
    \draw[subrib] ($(c3)!0.6!(btm1)$) -- ++(1.5, 0) node[item] {奥まで刺さってない};
    \draw[subrib] ($(c3)!0.4!(btm1)$) -- ++(1.5, 0) node[item] {部品を逆挿している};
    \draw[subrib] ($(c3)!0.2!(btm1)$) -- ++(1.5, 0) node[item] {接続・配線の誤り};

    % =============================================
    % ③ 部品(右上)
    % =============================================
    \draw[rib] (c2) -- (top2) node[above, catlabel] {③ 部品};
    \draw[subrib] ($(c2)!0.8!(top2)$) -- ++(1.5, 0) node[item] {LED/コンデンサ極性};
    \draw[subrib] ($(c2)!0.6!(top2)$) -- ++(1.5, 0) node[item] {ICのピン配置};
    \draw[subrib] ($(c2)!0.4!(top2)$) -- ++(1.5, 0) node[item] {抵抗値・容量の間違い};
    \draw[subrib] ($(c2)!0.2!(top2)$) -- ++(1.5, 0) node[item] {部品が違う};

    % =============================================
    % ④ 設計・知識(右下)
    % =============================================
    \draw[rib] (c4) -- (btm2) node[below, catlabel] {④ 設計・知識};
    \draw[subrib] ($(c4)!0.8!(btm2)$) -- ++(1.5, 0) node[item] {回路図を間違えている};
    \draw[subrib] ($(c4)!0.6!(btm2)$) -- ++(1.5, 0) node[item] {電流(が足らない)};
    \draw[subrib] ($(c4)!0.4!(btm2)$) -- ++(1.5, 0) node[item] {ホゲホゲ};
    \draw[subrib] ($(c4)!0.2!(btm2)$) -- ++(1.5, 0) node[item] {ホゲホゲ};


    % =============================================
    % ⑤ Rasberry Pi 
    % =============================================
    \draw[rib] (c5) -- (top3) node[above, catlabel] {⑤ プロセッサ};
    \draw[subrib] ($(c5)!0.8!(top3)$) -- ++(1.5, 0) node[item] {ソフトウェアが更新されてない};
    \draw[subrib] ($(c5)!0.6!(top3)$) -- ++(1.5, 0) node[item] {ソフトウェアのバグ};
    \draw[subrib] ($(c5)!0.4!(top3)$) -- ++(1.5, 0) node[item] {ピン指定が違う};
    \draw[subrib] ($(c5)!0.2!(btm2)$) -- ++(1.5, 0) node[item] {ホゲホゲ};



\end{tikzpicture}

%%
%% 更新日を記入する
%% 
\vfill
\hfill \small 更新日: \today 作成者:小泉直也


\end{document}
%\end{comment}
